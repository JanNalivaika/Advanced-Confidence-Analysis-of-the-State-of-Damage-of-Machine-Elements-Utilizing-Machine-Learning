%\textcolor{red}{Nomenklautur wird in \texttt{Nomenklatur.tex} zusammengefasst und automatisch eingebunden. So sieht es beispielhaft aus:}\footnote{Fußnutten sind generell zulässig, sollten aber nur in Ausnahmefällen verwendet werden.} 
\chapter{Nomenclature}
\begin{tabular}{llp{0.7\textwidth}}
\textbf{Zeichen} & \textbf{Einheit} & \textbf{Benennung}\\
$a$ & m & Achsabstand\\
$c_\gamma$ & $\text{N}/\mu\text{m}\cdot \text{mm}$ & Verzahnungssteifigkeit\\
$d_\mathrm{b}$ & mm & Grundkreisdurchmesser\\
\ldots
\end{tabular}
\vfill
%\textcolor{red}{!!! Hinweis !!! Nur Größe bzw. Variable darf kursiv geschrieben werden. Dies gilt auch für Indizes. So schreibt man $d_\mathrm{b}$ und nicht $d_b$ für Grundkreisdurchmesser. (Achte auf $\mathrm{b}$).}

%\textcolor{red}{Auch die Einheiten dürfen nicht kursiv geschrieben werden. Dokumentation dazu ist in FZGltx Doku Ordner zu finden in \texttt{typefaces.pdf}.}

%\textcolor{red}{Auch Einheiten werden ohne Klammer geschrieben, s. DIN 1313.}
%\vfill
%\begin{tabular}{lp{0.4\textwidth}lp{0.4\textwidth}}
%\multicolumn{4}{l}{\textbf{Indizes}}\\
%$\mathrm{1}$ & Ritzel bzw. unterer Wälzkörper & $\mathrm{2}$ & Ratzel bzw. oberer Wälzkörper\\
%$\mathrm{Bl}$ & Berührlinie & $\mathrm{dyn}$ & Dynamisch\\
%\ldots
%\end{tabular}
%\vfill
%\begin{tabular}{lp{0.4\textwidth}lp{0.4\textwidth}}
%\multicolumn{4}{l}{\textbf{Abkürzungen}}\\
%DZP & Däs Zöhnrad Progrömm & DIN & Deutsche Innovativste Nachzumachende Vorschriften\\
%\FzGeIl & \textbf{FZG e}lektronische \textbf{I}llustrationen & \FzGl & Spitzname von \FzGeIl\\

%\ldots
%\end{tabular}
%\vfill