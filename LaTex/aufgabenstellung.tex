Eine Vielzahl von Maschinenelementen ist derart dimensioniert, dass diese unter den Belastungen des vorgesehenen Anwendungsfalls eine endliche Lebensdauer aufweisen. Durch diese Herangehensweise können Masse und Volumen der Maschinenelemente minimiert und deren Herstellungskosten und CO2 – Fußabdruck reduziert werden.
Im Betrieb sind Maschinenelemente, wie zum Beispiel Zahnräder, oftmals keiner konstanten Belastung ausgesetzt, sondern deren Intensität variiert über die Betriebszeit. Um dies bei der Auslegung und der Nachrechnung zu berücksichtigen wird eine Schadensakkumulation durchgeführt. Aktuell ist hierbei der lineare Ansatz zur Schadensakkumulation am weitesten verbreitet und das durch internationale Normen vorgeschlagene Verfahren. Dieser Ansatz zeichnet sich insbesondere durch seine Simplizität und allgemeingültige Anwendbarkeit aus. Jedoch werden im Rahmen des Ansatzes eine Vielzahl von potenziellen Einflussgrößen auf die Lebensdauer bzw. den Schädigungszustand vernachlässigt, unter anderem die Reihenfolge von variierenden Belastungen. Dies kann zu einer verminderten Konfidenz der Ergebnisse und damit zu einer unzuverlässigeren Prognose der Restlebensdauer führen.
Der Einfluss der Reihenfolge der Belastung ist analytisch sehr komplex zu erfassen, da mit einer enormen Varianz an möglichen Verläufen umgegangen werden muss. Deshalb soll im Rahmen dieser Studienarbeit eine Möglichkeit erarbeitet werden Methoden des Maschinellen Lernens mit dem Ziel einzusetzen, dem Anwender eine Bewertung des Konfidenzbereichs einer bestimmten Restlebensdauer zu erleichtern. Basis der Bestimmung der Restlebensdauer ist immer zunächst der aktuelle Schädigungszustand, deshalb soll sich diese Arbeit auf die Bewertung von diesem beschränken.


Hierbei sind insbesondere folgende Arbeitspakete zu bearbeiten:

\begin{itemize}
   \item Einarbeitung in die Grundlagen des Maschinellen Lernens und der Betriebsfestigkeit
   \item Sichtung und Aufbereitung von bereitgestellten Versuchsdaten zur Lebensdauer von Zahnrädern
   \item Systematischer Entwurf eines Konzepts basierend auf Methoden des Maschinellen Lernens um die Konfidenz von linearen Schädigungszustandsberechnungen zu bewerten
   \item Aufbau eines Programms, das ausgehend von Belastungs-Zeit-Verläufen die Zuverlässigkeit von Schädigungszustandsberechnungen einschätzt (zum Beispiel mittels der Programmiersprache Python und geeigneter Bibliotheken)
   \item Aufbereitung und Einordnung der erzielten Ergebnisse
   \item Übersichtliche und vollständige Dokumentation des Stands der Technik, der Arbeitsschritte und der Ergebnisse
\end{itemize}