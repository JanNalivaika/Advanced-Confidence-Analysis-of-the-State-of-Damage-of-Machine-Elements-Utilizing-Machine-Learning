% !TeX spellcheck = en_US 
\chapter{Conclusion}\label{conc}
\section{Summary}
Machine elements depend on correct dimensioning to function as expected over their course of life. So far, the most prominent method for fatigue strength analysis is the Miner rule, which is accepted as an international standard. The downside of the Miner rule is that the order of the occurring loads is not taken into account. Regarding the current state of research, no theory has been developed that takes the order (history) of load into account and is easily transferable between machine elements.
The goal of this thesis is to develop a method that is able to provide a confidence value in the calculated state-of-damage by specifically considering the history of the loads that a machine element is subjected to. This thesis proposes a method that can classify a load sequence into one of three possible classes, depending on the effect of the load history. Further, a second indicator called state-of-health is added to analyze how much a machine element is worn down. With the combination of these two values, the state-of-damage is now supplied with more information and can be viewed from a different perspective. 
The provided load sequences are augmented with various techniques to form an enriched data-set. The analyzed classification methods return very promising results and are capable of correctly classifying a load sequence in 80~\% of cases. 
The selected regression models are also capable of predicting the state-of-health with very low error margins.
In conclusion, this method provides a very capable and robust alternative to nonlinear damage accumulation models that provides accurate results and is easily applicable to every machine element.  


\section{Outlook}
One of the most important aspects when it comes to ML approaches is the amount and quality of the data. To get an even more accurate result of the predicted classes and SOH, the supply of more data is the most important step. Further, it is possible to add more DA methods to increase the model's ability to generalize. 

Regarding the separation into classes, the proposed method only divides the load sequences into three distinct groups (Class -1, Class 0, Class 1). For a more precise prediction, it could be beneficial to increase the number of classes to get an understanding of how much the Miner rule is deviating from the predicted damaging effect of the load sequence.

Additionally, further models like RNNs can be implemented that can predict the class and SOH on the fly. This could be a form of condition monitoring that could give an even more accurate prediction and is also not dependent on iteratively retraining for a different damage sum D. Another option is the implementation of Bayesian Neural Networks that quantify the uncertainty in the predicted class. 

A further area of possible research is the prediction of the SOH not with a simple linear interpolation but with different functions that can depend on the output of the previous classification. Optionally, the label function can be a scaled Miner rule so that the label function has the value~1 at the end of the sequence and includes some knowledge about the Miner rule.  